% Remover as subseções e listar apenas as restrições

% Restrições

\begin{equation}
\sum_{n_1 \in \Omega_1(d)} y_{d,n_1} = 1, \quad \forall d \in D \tag{1}
\end{equation}

\begin{equation}
x_{d,n_1,p} = y_{d,n_1} \cdot dem_d \cdot \alpha_p, \quad \forall d \in D, n_1 \in \Omega_1(d), p \in P \tag{2}
\end{equation}

\begin{equation}
\sum_{n_2 \in \Omega_2(n_1)} x_{n_1,n_2,p} = \beta_{12} \cdot \sum_{d \in D: n_1 \in \Omega_1(d)} x_{d,n_1,p}, \quad \forall n_1 \in N_1, p \in P \tag{3}
\end{equation}

\begin{equation}
\sum_{n_3 \in \Omega_3(n_2)} x_{n_2,n_3,p} = \beta_{23} \cdot \sum_{n_1 \in N_1: n_2 \in \Omega_2(n_1)} x_{n_1,n_2,p}, \quad \forall n_2 \in N_2, p \in P \tag{4}
\end{equation}

\begin{equation}
\sum_{d \in D: n_1 \in \Omega_1(d)} \sum_{p \in P} x_{d,n_1,p} \leq \left(\sum_{d \in D} dem_d\right) \cdot z_{n_1}, \quad \forall n_1 \in N_1^{cand} \tag{5}
\end{equation}

\begin{equation}
\sum_{n_1 \in N_1: n_2 \in \Omega_2(n_1)} \sum_{p \in P} x_{n_1,n_2,p} \leq \left(\sum_{d \in D} dem_d\right) \cdot z_{n_2}, \quad \forall n_2 \in N_2^{cand} \tag{6}
\end{equation}

\begin{equation}
\sum_{n_2 \in N_2: n_3 \in \Omega_3(n_2)} \sum_{p \in P} x_{n_2,n_3,p} \leq \left(\sum_{d \in D} dem_d\right) \cdot z_{n_3}, \quad \forall n_3 \in N_3^{cand} \tag{7}
\end{equation}

\begin{equation}
\sum_{d \in D: n_1 \in \Omega_1(d)} \sum_{p \in P} x_{d,n_1,p} \leq Cap_1, \quad \forall n_1 \in N_1 \tag{8}
\end{equation}

\begin{equation}
\sum_{n_1 \in N_1: n_2 \in \Omega_2(n_1)} \sum_{p \in P} x_{n_1,n_2,p} \leq Cap_2, \quad \forall n_2 \in N_2 \tag{9}
\end{equation}

\begin{equation}
\sum_{n_2 \in N_2: n_3 \in \Omega_3(n_2)} \sum_{p \in P} x_{n_2,n_3,p} \leq Cap_3, \quad \forall n_3 \in N_3 \tag{10}
\end{equation}

\begin{equation}
CNES_{n_1,e_1} + eq_{e_1,n_1} = \sum_{d \in D: n_1 \in \Omega_1(d)} \sum_{p \in P} x_{d,n_1,p} \cdot cap_{e_1}, \quad \forall e_1 \in E_1, n_1 \in N_1^{real} \tag{11}
\end{equation}

\begin{equation}
eq_{e_1,n_1} = \sum_{d \in D: n_1 \in \Omega_1(d)} \sum_{p \in P} x_{d,n_1,p} \cdot cap_{e_1}, \quad \forall e_1 \in E_1, n_1 \in N_1^{cand} \tag{12}
\end{equation}

\begin{equation}
CNES_{n_2,e_2} + eq_{e_2,n_2} = \sum_{n_1 \in N_1: n_2 \in \Omega_2(n_1)} \sum_{p \in P} x_{n_1,n_2,p} \cdot cap_{e_2}, \quad \forall e_2 \in E_2, n_2 \in N_2^{real} \tag{13}
\end{equation}

\begin{equation}
eq_{e_2,n_2} = \sum_{n_1 \in N_1: n_2 \in \Omega_2(n_1)} \sum_{p \in P} x_{n_1,n_2,p} \cdot cap_{e_2}, \quad \forall e_2 \in E_2, n_2 \in N_2^{cand} \tag{14}
\end{equation}

\begin{equation}
CNES_{n_3,e_3} + eq_{e_3,n_3} = \sum_{n_2 \in N_2: n_3 \in \Omega_3(n_2)} \sum_{p \in P} x_{n_2,n_3,p} \cdot cap_{e_3}, \quad \forall e_3 \in E_3, n_3 \in N_3^{real} \tag{15}
\end{equation}

\begin{equation}
eq_{e_3,n_3} = \sum_{n_2 \in N_2: n_3 \in \Omega_3(n_2)} \sum_{p \in P} x_{n_2,n_3,p} \cdot cap_{e_3}, \quad \forall e_3 \in E_3, n_3 \in N_3^{cand} \tag{16}
\end{equation}

\begin{equation}
z_{n_1} = 1, \quad \forall n_1 \in N_1^{real} \tag{17}
\end{equation}

\begin{equation}
z_{n_2} = 1, \quad \forall n_2 \in N_2^{real} \tag{18}
\end{equation}

\begin{equation}
z_{n_3} = 1, \quad \forall n_3 \in N_3^{real} \tag{19}
\end{equation}

% Parágrafo explicativo das restrições

As restrições (1) e (2) garantem que toda demanda seja alocada a uma única unidade primária e que o fluxo de pacientes seja consistente com a alocação e o perfil populacional. As restrições (3) e (4) modelam o encaminhamento de pacientes para os níveis secundário e terciário, respeitando os percentuais de referência. As restrições (5)-(7) asseguram que apenas unidades abertas possam receber pacientes, limitando o fluxo conforme a decisão de abertura. As restrições (8)-(10) impõem os limites de capacidade das unidades em cada nível. As restrições (11)-(16) garantem que a quantidade de equipes seja suficiente para atender o fluxo de pacientes, considerando equipes existentes e novas. Por fim, as restrições (17)-(19) fixam a abertura das unidades já existentes no sistema. 